\chapter{Стратегии масштабирования для PostgreSQL}
\begin{epigraphs}
\qitem{То, что мы называем замыслом (стратегией), означает избежать бедствия и получить выгоду.}{У-цзы}
\qitem{В конце концов, все решают люди, не стратегии.}{Ларри Боссиди}
\end{epigraphs}

\section{Введение}
Многие разработчики крупных проектов сталкиваются с проблемой, когда один единственный сервер 
базы данных никак не может справится с нагрузками. Очень часто такие проблемы происходят из-за 
неверного проектирования приложения(плохая структура БД для приложения, отсутствие кеширования). Но в данном 
случае пусть у нас есть <<идеальное>> приложение, для которого оптимизированы все SQL запросы, используется кеширование, 
PostgreSQL настроен, но все равно не справляется с нагрузкой. Такая проблема может возникнуть как на этапе проектирования, 
так и на этапе роста приложения. И тут возникает вопрос: какую стратегию выбрать при возникновении подобной ситуации?

Если Ваш заказчик готов купить супер сервер за несколько тысяч долларов 
(а по мере роста~--- десятков тисяч и т.д.), чтобы сэкономить время разработчиков, но сделать все быстро, 
можете дальше эту главу не читать. Но такой заказчик~--- мифическое существо и, в основном, такая проблема 
ложится на плечи разработчиков.

\subsection{Суть проблемы}

Для того, что-бы сделать какой-то выбор, необходимо знать суть проблемы. 
Существуют два предела, в которые могут уткнуться сервера баз данных:

\begin{itemize}
\item Ограничение пропускной способности чтения данных;
\item Ограничение пропускной способности записи данных;
\end{itemize}

Практически никогда не возникает одновременно две проблемы, по крайне мере, это маловероятно (если вы конечно не Twitter 
или Facebook пишете). Если вдруг такое происходит~--- возможно система неверно спроектирована, и её реализацию следует пересмотреть.