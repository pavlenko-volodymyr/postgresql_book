\thispagestyle{empty}
При написании книги(мануала, или просто шпаргалки) использовались материалы:
\begin{itemize}
\item PostgreSQL: настройка производительности. Алексей Борзов (Sad Spirit) borz\_off@cs.msu.su, \\
http://www.phpclub.ru/detail/store/pdf/postgresql-performance.pdf
\item Настройка репликации в PostgreSQL с помощью системы Slony-I, Eugene Kuzin eugene@kuzin.net, \\
http://www.kuzin.net/work/sloniki-privet.html
\item Установка Londiste в подробностях, Sergey Konoplev gray.ru@gmail.com, \\
http://gray-hemp.blogspot.com/2010/04/londiste.html
\item Учебное руководство по pgpool-II, Dmitry Stasyuk, \\
http://undenied.ru/2009/03/04/uchebnoe-rukovodstvo-po-pgpool-ii/
\item Горизонтальное масштабирование PostgreSQL с помощью PL/Proxy, Чиркин Дима dmitry.chirkin@gmail.com, \\
http://habrahabr.ru/blogs/postgresql/45475/
\item Hadoop, Иван Блинков wordpress@insight-it.ru, \\
http://www.insight-it.ru/masshtabiruemost/hadoop/
\item Up and Running with HadoopDB, Padraig O'Sullivan, \\
http://posulliv.github.com/2010/05/10/hadoopdb-mysql.html
\item Масштабирование PostgreSQL: готовые решения от Skype, Иван Золотухин, \\
http://postgresmen.ru/articles/view/25
\item Streaming Replication, \\
http://wiki.postgresql.org/wiki/Streaming\_Replication
\item Шардинг, партиционирование, репликация - зачем и когда?, Den Golotyuk, \\
http://highload.com.ua/index.php/2009/05/06/шардинг-партиционирование-репликац/
\end{itemize}


\clearpage